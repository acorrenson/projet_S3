\documentclass[11pt]{article}
\usepackage[utf8]{inputenc}
\usepackage{lmodern}
\usepackage[T1]{fontenc}
\usepackage{hyperref}
\usepackage{graphicx}
\usepackage{amsfonts}

\hypersetup{
  % parametrage des hyperliens
  colorlinks = true, % colorise les liens
  breaklinks = true, % permet les retours à la ligne pour les liens trop longs
  linkcolor = black, % couleur des liens internes aux documents
  citecolor = black, % couleur des liens vers les references bibliographiques
  urlcolor = blue, % couleur des hyperliens
}
\begin{document}

\title{Bilan du Projet S3}
\author{Arthur Correnson}
\maketitle

\section*{Introduction}

Lors du semestre 3 de licence, il nous a été demandé d'implémenter en langage C un logiciel de résolution du \textbf{problème du voyageur de commerce}. Ce problème d'optimisation combinatoire est insoluble en grande dimension, mais plusieurs méthodes permettent d'approcher la solution optimale à moindre coût. Nous proposons ici quatre méthodes de résolution :

\begin{itemize}
	\item \textbf{Force Brute} 	Recherche exhaustive du plus court chemin.
	\item \textbf{Algorithme génétique} 	Approche évolutive basée sur le croisement de solutions intermédiaires.
	\item \textbf{Plus Proche voisin}	 	Méthode heuristique calculant de proche en proche le plus court chemin.
	\item \textbf{Marche aléatoire} Tirage au sort d'un chemin.
\end{itemize}

Notons qu'une méthode d'optimisation nommée \textbf{2 optimisation} a aussi été développée. Elle permet d'améliorer les résultats obtenus par les méthodes précédentes.

\section*{Programme rendu}

La réponse au sujet prend la forme d'un programme C utilisable en ligne de commande. Il attend en entrée un fichier au format standard \href{http://comopt.ifi.uni-heidelberg.de/software/TSPLIB95/}{TSP} décrivant le problème à résoudre et produit une solution au problème sous la forme d'un fichier \textbf{CSV}. Les quatre méthodes demandées ont été implémentées et vérifiée.

\subsection*{Structure des sources}

Le code source est organisé selon une arborescence stricte :

\begin{itemize}
	\item 	\textbf{docs/} : La documentation
	\item 	\textbf{scripts/} : Scripts utiles python (affichage graphique des résultats)
	\item 	\textbf{data/} : Exemples de fichiers TSP et des résultats associés.
    	\item 	\textbf{src/} : contient le code source en C
    \item 	\textbf{src/tsplib/} : module d'interaction avec les fichiers au format TSP.
	\item	\textbf{src/methods/} : implémentation des différentes méthodes de résolution
   	\item 	\textbf{include/} : contient tous les en-têtes
   	\item	\textbf{include/tsplib} : en-têtes pour le module tsplib
	\item 	\textbf{include/methods} : en-têtes pour le méthodes de résolution
    \item 	\textbf{tests/} : dossier contenant les modules de tests unitaires.
\end{itemize}

\subsection*{Compilation}


Notons que la séparation des sources et des en-têtes est un choix de design utilisé afin de faciliter l'usage de notre programme à la fois en tant que logicielle, mais également en tant que bibliothèque utilitaire. Le programmeur souhaitant re-exploiter tout ou partie de ces sources pourra aisément importer les en-têtes de son choix en passant l'option -I include/ au compilateur C.

\subsection*{Utilisation du programme}

\begin{tabular}{ | c | l |}
	\textbf{option}	& \textbf{méthode} \\
	-bf		& Force Brute \\
	-bfm	& Force Brute (optimisation matricielle) \\
	-ga		& algorithme génétique \\
	-rw		& marche aléatoire (\textit{Random Walk}) \\
	-ppv		& plus proche voisin (\textit{Nearest Neighbour}) \\
	-2opt 	& application de la \textbf{2-optimisation} \\
\end{tabular}

\end{document}